
%\hypertarget{aula-9}{%
%\chapter{Aula: 9}\label{aula-9}}




\hypertarget{camada-de-aplicauxe7uxe3o-a-razuxe3o-de-existir-das-redes}{%
\chapter{Camada de Aplicação: a razão de existir das redes}\label{camada-de-aplicauxe7uxe3o-a-razuxe3o-de-existir-das-redes}}

A camada de aplicação (primeira camada) é considerada como a mais
importante, pois ela é quem gera a utilidade da rede. Assim, a
infraestrutura, bem como os protocolos, foram criados para servi-la, e,
portanto, sem a aplicação, não existiria as demais camadas. Essa
utilidade vem dos benefícios produzidos pela interação de dois processos
executados em computadores diferentes e, possivelmente, sistemas
operacionais diferentes. Para tal, há duas principais arquiteturas,
\emph{client-server} e \emph{peer-to-peer} (P2P).

\hypertarget{http}{%
\section{HTTP}\label{http}}

A arquitetura \emph{client-server} é empregue na aplicação \emph{Web}, o
qual usufrui do protocolo \emph{Hypertext Transfer Protocol} (HTTP).
Essa arquitetura funciona no sistema \emph{always on}, no qual o
servidor, ou processo que espera para ser contactado no início da
interação, deve sempre ficar ativo, e o seu endereço na rede fixado. Os
clientes não interagem diretamente entre si. O protocolo de transporte
usado é o TCP.

A escalabilidade dessa arquitetura (a capacidade de suportar o aumento
do número de clientes) está limitada pela capacidade da infraestrutura
já instalada. Assim, o aumento no número de \emph{requests} (derivado do
aumento no número de \emph{clients}) deve ser precedido por um aumento
na competência do servidor de lidar com os mesmos. Por essa razão, os
servidores são comumente hospedados em \emph{data centers}, algo que
facilita a escalabilidade, por alocar, dinamicamente, centenas de
dispositivos para o processo \emph{server} (o servidor se torna uma
entidade processada por múltiplos dispositivos).

O HTTP define a estrutura das mensagens e a forma que as mesmas são
trocadas. Essa troca é baseada no padrão \emph{response-request} (a
mensagem do \emph{client} é o \emph{request}, e a reposta do
\emph{server} o \emph{response}). O \emph{client} (uma navegador como o
\emph{Chrome}, por exemplo), definido por ser o processo que inicia a
comunicação, necessita do endereço do \emph{server}, chamado de URL, que
é composta por dois componentes, o \emph{hostname} (
\emph{http://www.google.com/}) e o \emph{pathname}
(\emph{/search?q=alguma+coisa}):

\begin{verbatim}
URL: http://www.google.com/search?q=alguma+coisa
\end{verbatim}

A mensagem HTTP \emph{request} segue o seguinte padrão:

\emph{Request Line}:
\texttt{GET\ /somedir/page.html\ HTTP/1.1\ \textbackslash{}r\textbackslash{}n}
(= Método + URL + HTTP \emph{version} + \emph{carriage return and line
feed})\\
\emph{Header lines} (line 1):
\texttt{HOST:\ www.google.com.br\ \textbackslash{}r\textbackslash{}n}
(\emph{host} no qual hospeda o objeto requisitado)\\
\emph{Header lines} (line 2):
\texttt{Connection:\ Mozilla/5.0\ \textbackslash{}r\textbackslash{}n}
(navegador \emph{Firefox})\\
\emph{Header lines} (line 3):
\texttt{Accept-language:\ fr\ \textbackslash{}r\textbackslash{}n}
(prefere a versão em língua francesa)\\
\emph{Header lines} (line x):
\texttt{...\ \textbackslash{}r\textbackslash{}n} (outras
configurações)\\
\emph{Entity body} (line y): \texttt{\textbackslash{}r\textbackslash{}n}
(pode ficar vazio, como no método \emph{GET}, ou preenchido, como no
método \emph{POST})

A mensagem HTTP \emph{response} segue o seguinte padrão:

\emph{Status Line}:
\texttt{HTTP/1.1\ 200\ OK\ \textbackslash{}r\textbackslash{}n} (Tudo
ocorreu bem)\\
\emph{Header lines} (line 1):
\texttt{Connection:\ close\ \textbackslash{}r\textbackslash{}n}
(\emph{server} irá encerrar a conexão)\\
\emph{Header lines} (line 2):
\texttt{Date:\ Tue,\ 18\ Aug\ 2015\ 15:44:04\ GMT\ \textbackslash{}r\textbackslash{}n}
(data em que o \emph{response} foi enviado)\\
\emph{Header lines} (line 3):
\texttt{Server:\ Apache/2.2.3\ \textbackslash{}r\textbackslash{}n} (nome
do servidor)\\
\emph{Header lines} (line 4):
\texttt{Last-Modified:\ Tue,\ 18\ Aud\ 2015\ 15:11:03\ GMT} (data da
última modificação)\\
\emph{Header lines} (line 5):
\texttt{Content-Length:\ 6821\ \textbackslash{}r\textbackslash{}n}
(número de bytes do objeto enviado - O mesmo está em \emph{Entity
body})\\
\emph{Header lines} (line 6):
\texttt{Content-Type:\ text/html\ \textbackslash{}r\textbackslash{}n}
(indica o formato do objeto, no caso uma página html)\\
\emph{Entity body} (line y):
\texttt{...\ \textbackslash{}r\textbackslash{}n} (dados do objeto)

\hypertarget{torrent}{%
\section{Torrent}\label{torrent}}

A segunda arquitetura citada anteriormente, a P2P, baseia-se na
inter-colaboração intermitente dos dispositivos na rede, ou \emph{peers},
de forma que esses podem ser \emph{client} e \emph{server}
simultaneamente (É importante perceber que, em uma sessão de comunicação
com um par de dispositivos, o \emph{client} é aquele que inicia o
contato e o \emph{server} aquele que espera para ser contactado, assim,
apesar de cada \emph{peer} assumir ambas as formas simultaneamente nessa
rede, elas tem um único papel para cada sessão). Essa arquitetura é
auto-escalável (\emph{self-escalability}), pois cada \emph{peer}, apesar
de adicionar novos \emph{requests}, adicionam também capacidade de
serviço na rede.

A popularidade desse protocolo vem do uso da tecnologia \emph{Torrent}.
Desenvolvida pela empresa \emph{BitTorrent}, essa tecnologia
descentraliza o fornecimento de arquivos, de forma que os \emph{peers},
conforme fazem o \emph{download}, também proporcionam os dados já
baixados de volta para a rede. Assim, a capacidade de transmissão não
está limitada pelo \emph{seed} (primeiro fornecedor de um conteúdo),
crescendo, e, consequentemente, tornando arquivos populares mais
acessíveis, conforme aumenta a participação dos usuários (sendo,
consequentemente, barato, por não necessitar a contratação de uma
infraestrutura para o fornecimento de arquivos).

Essa arquitetura enfrenta diversos desafios, como: segurança, performance
e confiabilidade.

\hypertarget{sockets}{%
\section{Sockets}\label{sockets}}

Os \emph{sockets} são uma estrutura de dados em \emph{kernel space}
(encontrados dentro do sistema operacional, ou seja, seu acesso é
através de uma chamada de sistema) que fornece (analogamente a uma
porta) uma interface (API, \emph{Application Programming Interface})
para a interação da camada de aplicação com a camada de transporte. Ele
se manifesta através de um FD (\emph{file descriptor}, um arquivo),
fornecendo um \emph{pipeline} para o processo (em \emph{user-space}).
Para ser executado, o \emph{socket} necessita do endereço do \emph{host}
(\emph{internet protocol address} ou \emph{IP address}) e do
identificador que especifica o processo receptor no \emph{host} (chamado
de \emph{Port Number}). Por padrão, o \emph{web server} utiliza a porta
80, e o \emph{mail server} (com o protocolo SMTP) é identificado pela
porta 25.

\hypertarget{serviuxe7os-de-transporte}{%
\section{Serviços de transporte}\label{serviuxe7os-de-transporte}}

Como citado anteriormente, a camada de transporte fornece uma série de
serviços para a camada superior. Esses serviços podem ser analisados sob
a ótica de: confiabilidade (\emph{reliable data transfer}), taxa de
transferência (\emph{throughput}), atraso (\emph{delay} ou \emph{timing})
e segurança (\emph{security}). Nesse sentido, a \emph{internet} prover
dois protocolos de transporte, o UDP e o TCP, os quais disponibiliza
serviços com características diferentes, de maneira que:

TCP:

\begin{enumerate}
\def\labelenumi{\arabic{enumi}.}
\tightlist
\item
  \emph{Handshaking procedure}: com uma preparação anterior ao envio da
  mensagem (confiabilidade).
\item
  \emph{Full duplex connection}: com os processos podendo receber e
  enviar dados.
\item
  \emph{Reliable data transfer service}: sendo os dados enviados sem
  erros e na ordem correta (confiabilidade).
\item
  \emph{Transport Layer Secutiry: uma melhora no TCP que fornece
  criptografia (}encryption\emph{), integridade de dados (}data
  integrity\emph{) e }end-point authentication* (segurança).
\item
  Garantia de entrega da mensagem.
\end{enumerate}

UDP (um \emph{lightweight transport protocol}):

\begin{enumerate}
\def\labelenumi{\arabic{enumi}.}
\tightlist
\item
  \emph{No handshaking procedure}: Não se sabe se o receptor está
  preparado para receber a mensagem.
\item
  \emph{Unreliable data transfer}: podendo ser recebidos dados
  corrompidos e fora de ordem.
\item
  Não há garantia de entrega da mensagem.
\item
  Rápida.
\end{enumerate}

É importante perceber que os serviços de \emph{timing} e
\emph{throughput}, apesar de serem necessários para algumas aplicações,
como \emph{internet telephony} (Skype), não são fornecidos pelos
protocolos de \emph{internet} atuais.

\hypertarget{complementar-3}{%
\section{Complementar}\label{complementar-3}}

\hypertarget{pesquisar-sobre-2}{%
\subsubsection{Pesquisar Sobre}\label{pesquisar-sobre-2}}

\begin{enumerate}
\def\labelenumi{\arabic{enumi}.}
\tightlist
\item
  \emph{Socket programming}: página 152 do livro \emph{Computer
  Networking a top-down approach 8th ed.}
\item
  \emph{Dual Stack}: IPv4 e IPv6
\item
  NAT
\item
  WireShark: www.wireshark.org
\item
  ip addr shwo
\item
  ip config /on
\item
  telnet 127.0.0.1 3000
\item
  SS - apt \textbar{} grep 3000
\item
  SS - apt \textbar{} grep telnet
\item
  nc - V - U 127.0.0.1 1200
\item
  ncat
\item
  CDN (\emph{Content Distribution Networks})
\item
  DASH (\emph{Dynamic Adaptive Streaming over HTTP}).
\item
  DNS (\emph{domain name system}): The Internet's Directory Service
\item
  telnet serverName 25
\item
  SMTP (\emph{Simple Mail Transfer Protocol})
\item
  \emph{Cookies} (\emph{User interaction})
\end{enumerate}
