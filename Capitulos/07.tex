
%\hypertarget{aula-7---redes.}{%
%\chapter{Aula: 7}\label{aula-7}}



\hypertarget{introduuxe7uxe3o-uxe0-redes-de-computadores}{%
\chapter{Introdução à redes de
computadores}\label{introduuxe7uxe3o-uxe0-redes-de-computadores}}

A Internet é uma rede mundial de computadores que interconecta, por meio
de \emph{communication links}, bilhões de dispositivos \emph{hosts}
(anfitriões), também referidos como \emph{end systems} (sistemas finais)
por estarem na borda de internet, como os clientes (computadores
pessoais) e servidores. Seu funcionamento é análogo ao modal de
transporte rodoviário, pois assim como um produto, que ao sair de uma
manufatura, deve ser empacotado, carregado no caminhão, transportado
através das rodovias, descarregado no destino, e desempacotado, para em
fim estar disponível para uso, os dados também passam pelo mesmo
processo, pois ao sair de um \emph{host}, devem ser empacotados,
carregados no protocolo de transmissão (como \emph{Transmission Control
Protocol} e \emph{Internet Protocol}, que definem o formato, utilizando
o \emph{struct} na linguagem \texttt{c}, e a ordem das mensagens),
transportados por um meio físico (como cabos ou espectro
eletromagnético), descarregados do protocolo no destino, e
desempacotados, para finalmente serem usados por alguma aplicação.

É importante citar que os protocolos de Internet são desenvolvidos pela
\emph{Internet Engineering Task Force} (IETF) e os seus documentos são
chamados de \emph{requests for comments} (RFCs).

Essa rede pode também ser definida como uma plataforma que oferece
serviços de comunicação entre aplicações, tornando-se, dessa forma,
análogo à um sistema de correios que oferece serviços aos seus clientes
como entrega normal ou expressa.

O acesso à internet é fornecido por ISPs (\emph{Internet Service
Providers), que transmitem os dados de forma guiada (}guided
media\emph{), através, por exemplo, de cabos de cobre de par trançado,
redes de telefone (com o DSL, }Digital Subscriber Line\emph{), cabos de
televisão ou fibra optica (com o conceito }fiber to the home\emph{, ou
FTTH), ou de forma não guiada (}unguided media*), no qual ondas
eletromagnéticas propagam-se pela atmosfera e espaço, com o uso das
torres de rádio e dos satélites (como os geoestacionários, que
introduzem um atraso de 280 milissegundos na comunicação, e os de baixa
órbita).
